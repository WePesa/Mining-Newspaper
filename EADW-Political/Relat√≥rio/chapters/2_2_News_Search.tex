%!TEX encoding = UTF-8 Unicode
\subsection{Pesquisa de Notícias}
\label{sec:news_search}
\hspace{15pt}Despois da obtenção do artigo do seu sumario e titulo procedemos a indexaçao desta informação. A indexação é efectuada utilizando a ferramenta \textbf{Whoosh} utilizando o modelo \textbf{BM25}. Para diferenciar e modificar o pesso de cada parte da messagem por forma a obter melhores resultados utilizamos a seguinte metodologia: O ID sera o URL original do Artigo que é o nosso identificador unico no Sistema. E para cada par \textbf{(ID, TEXTO)} que introduzimos no Whoosh a parte do texto sera composta pelas tres partes do artigo de seguinte modo.\\\\
\centerline{\textbf{TEXTO = 10 x Título + 2 x Sumário + Artigo}}
\newline\newline
Isto permite-nos obter melhores resultados pois em testes efectuados verificamos que a informação relevante vem incluida no Título ou no Sumário do Artigo. \newline
\hspace{15pt}O Resultado de uma pesquisa é um conjunto de links por ordem crescente de relevancia. Para cada link efectuamos varias \textit{queries} a base de dados por forma a encontrar toda a informação estruturada que indexamos anteriormente. Logo como temos o as entidades presentes na base de dados para cada noticia apresentamos os 7 Entidades mais relevantes. A relevancia das entidades e dada em função de dois factores: o numero de vezes que a entidade aparece na noticia e a relevancia deste na lista de entidades fornecida para o projecto. Para cada entidade apresentamos o seu sentimento em releção a noticia, como tambem o sentimento geral da noticia.