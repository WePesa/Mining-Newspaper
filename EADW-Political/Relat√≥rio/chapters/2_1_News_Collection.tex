\subsection{Colecção e Armazenamento de Notícias}
\label{sec:news_storage}
\hspace{15pt}A recolha de noticias é iníciada descarregando os feeds dos prinicípais jornais politicos portugueses: Diário de Notícias, Jornal de Notícias, Diário Económico, Sol. Para cada um destes feeds utilizando a faramenta \textbf{feedparser}, extraimos o URL da notícia original e a data do feed. Estes dados são guardados na base de dados.\\
Imediatamente a seguir ou à posteriori, descarregamos o contéudo da notícia através do URL. Para tal utilizando a farramenta \textbf{Beautiful Soup}. Construímos um HTML parser para cada um dos websites. O contéudo é separado em Título, Sumário e Artigo de modo a atribuírmos relevância distinta a cada parte do texto. A entrada na base de dados é actualizada com o contéudo e marcado como processado.\\
Estes dados são então utilizados para realizar as pesquisas e para extracção de informação, assuntos abordados nas próximas secções.\\
Criamos tambem um sistema de caching das paginas descaregadas. Ao longo do projecto verificamos que exestia um bottleneck na extração da informação da Web, logo para acelerar o processo da aquisição da informação guardamos as paginas em disco utilizando varias threads e depois efectuamos o seu processamento. Para distinguir os ficheros, o seu nome é dado em função de uma função de resumo (SHA-1) do URL original.