%!TEX encoding = UTF-8 Unicode
\subsection{Analise do Sentimento}
\label{sec:sentiment_analysis}
\hspace{15pt}A análise de sentimento permite determinar a reputação e ter um feedback em tempo real do panorama político nacional. A opinião que os média publicam sobre um determinado político ou partido político é extremamente relevante e decisiva para o futuro político-partidário de um país. Em 2013, as eleições em Itália sofreram uma forte influência por parte dos média. A maioria dos média televisivos italianos são propriedade de Silvio Berlusconi, então candidato à liderança, fizeram campanhas de opinião positiva ao político e fazendo com que a opinião de um político envolvido em vários escândalos se altera-se e deste modo obtivesse quase o mesmo número de votos que Luigi Bersani.\newline\hspace{15pt}

\subsubsection{Análise do Sentimento ao nivel da Frase}
\hspace{15pt} Com vista à classificação do sentimento de cada entidade, temos primeiro de classificar a opinião da frase em que a entidade  é 
referida.\\
 A opinião das frases é obtida através dos seus adjectivos. Para identificar o sentimento dos adjectivos utilizamos um lexicon \textbf{(SentiLex-PT\_03)}, fornecido pelo projecto \textit{DMIR} do \textit{indesc-id} e armazenada na nossa base de dados \textit{sqlite}. Para classificar uma determinada frase verificamos todas as palavras da frase através da nossa base de dados  e, caso sejam adjectivo, a palavras como positiva ou negativa. Para cada frase é realizada a soma da quantidade de adjectivos positivos e negativos obtidos 
 e ponderada através da divisão pelo número total de palavras da frase porque 
 uma frase com mais adjectivos não dá forçosamente uma opinião mais negativa ou 
 mais positiva.  Através do mecanismo da secção anterior, extraimos as entidades de cada frase. Assumimos que existe apenas 1 opinião por cada frase 
 e associamos o sentimento da frase a esse político. \\
Uma das maiores dificuldades, especialmente em notícias políticas, é a análise de opiniões contídas nas citações de outros políticos. Estas citações são muitas vezes sarcásticas e difíceis de analisar. A frase é objetiva ou subjetiva, isto é, a frase contém alguma opinião, visão ou crença subjacente? Apenas consideramos frases objetivas porque frases subjetivas tem uma análise muito mais complexa que não vamos abordar neste projeto. 

\subsubsection{Análise do Sentimento ao nível do Documento}
\hspace{15pt}Admitimos um método de análise de \textit{Supervied Learning} em que classificamos o documento em 3 classes: Positivo, Negativo e Neutro. Para definir o sentimento do texto procedemos à soma dos sentimentos de todas as frases, obtendo a classificação do documento. Esta informação é útil para o caso de pesquisas mais especificas em que indicamos se queremos obter textos com sentimento positivo ou negativo.

