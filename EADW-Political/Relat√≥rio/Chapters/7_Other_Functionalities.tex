%!TEX encoding = UTF-8 Unicode
\subsection{Other Functionalities}
Realizámos a recolha de nomes de cada um dos ministros do actual governo e marcámos as entidades como membros do governo. Além disso, fizemos parsing ao site da assembleia da republica e recolhemos todos os nomes de deputados e respectivos partidos e associamos às respectivas entidades.\\
Com base nesta contextualização das entidades, realizámos a análises entre partidos e da avaliação do governo.
\subsubsection{Análise Partidária}


\subsubsection{Avaliação do Governo}


\subsubsection{Caracterização das entidades}
Recolhemos em cada frase os adjectivos que caracterizam as entidades dessa frase e associámos estes adjectivos à entidade. Deste modo, não só sabemos a opinião como também quais as caracteristicas mais faladas a cada uma das entidades.

\subsubsection{Evolução da opinião}
Para cada entidade, podemos consultar o número de noticias positivas e o nº de noticias negativas e deste modo saber como é que a popularidade da entidade evoluiu ao longo do tempo.

\subsubsection{Interface de Consulta}
A percepção sobre os dados recolhidos aumenta quando analisádos gráficamente. Para tal, criámos uma interface web de pesquisa. Esta interface foi criada no servidor Python Bottle \cite{bottle}. 